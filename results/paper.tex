\documentclass{amsart}
\newtheorem{thm}{Theorem}[section]
\newtheorem{prop}[thm]{Proposition}
\newtheorem{lem}[thm]{Lemma}
\newtheorem{cor}[thm]{Corollary}
\theoremstyle{definition}
\newtheorem{definition}[thm]{Definition}
\newtheorem{example}[thm]{Example}
\numberwithin{equation}{section}


\begin{document}


\title{Classification of Finite Symmetric Biquandles}

\author{Joe Nelson}
\email{joe@begriffs.com}
\urladdr{https://github.com/begriffs/semiquandle}
\author{Scott Pellicane}
\email{second@math.sc.edu}


\begin{abstract}
The goal is to classify finite symmetric biquandles, or semiquandles, first defined in [1].

We present an infinite class of simple semiquandles of prime order.
\end{abstract}

\maketitle

%\tableofcontents


%%%%%%%%%%%%%%%%%%%%%%%%%%%%%%%%%%%%%%%%%%%%%%%%%%%%%%%%%%%%%%%%%%%%%%
\section{Simple Semiquandles}
%%%%%%%%%%%%%%%%%%%%%%%%%%%%%%%%%%%%%%%%%%%%%%%%%%%%%%%%%%%%%%%%%%%%%%

\begin{thm}
Let $p$ be prime and let $\phi$ be a p-cycle in the symmetric group $S_p$.
Then the semiquandle $C(\phi)$ is simple.
\end{thm}
\begin{proof}
Let $\Theta$ be a congruence on $C(\phi)$. Suppose $\Theta$ has a class with
at least two elements, $a$ and $b$. We will show that this class is all of
$C(\phi)$, and hence that $\Theta$ is trivial.

Since $\phi$ is a cycle, $b = \phi^{i}(a)$ for some $i$; hence
$\phi^{i}(a) \equiv \phi^{0}(a)$. Since $\Theta$ is a congruence $\phi^{i}(a) =
\phi^{i}(\phi^{0}(a)) \equiv \phi^{i}(\phi^{i}(a)) = \phi^{2i}(a)$. By transitivity,
$a \equiv \phi^{2i}(a)$. Repeating the process gives $a \equiv \phi^{ni}(a)$ for any $n$.

If $c$ is an arbitrary element of $C(\phi)$, then $c = \phi^{j}(a)$ for some $j$.
Since the $\phi$ has prime order in $S_p$, there exists an $n$ with
$\phi^{j} = \phi^{ni}$ so $a \equiv \phi^{ni}(a) = c$.
\end{proof}


%\subsection{Things that need to be done.}

\end{document}
