\documentclass{amsart}
\newtheorem{thm}{Theorem}[section]
\newtheorem{prop}[thm]{Proposition}
\newtheorem{lem}[thm]{Lemma}
\newtheorem{cor}[thm]{Corollary}
\theoremstyle{definition}
\newtheorem{definition}[thm]{Definition}
\newtheorem{example}[thm]{Example}
\numberwithin{equation}{section}


\begin{document}


\title{Classification of Finite Symmetric Biquandles}

\author{Joe Nelson}
\author{Scott Pellicane}

\begin{abstract}
The goal is to classify finite symmetric biquandles, or semiquandles,
first defined in \cite{aH10}.

We present an infinite class of simple semiquandles of prime order.
\end{abstract}

\maketitle

%\tableofcontents

%%%%%%%%%%%%%%%%%%%%%%%%%%%%%%%%%%%%%%%%%%%%%%%%%%%%%%%%%%%%%%%%%%%%%%
\section{Definition of a Semiquandle}
%%%%%%%%%%%%%%%%%%%%%%%%%%%%%%%%%%%%%%%%%%%%%%%%%%%%%%%%%%%%%%%%%%%%%%

A \emph{semiquandle} (as defined in \cite{aH10}) is a system
$(Q,\nearrow,\searrow)$ such that $Q$ is a nonempty set and $\nearrow$
and $\searrow$ are binary functions on $Q$ satisfying:

\begin{enumerate}
\item $\exists!{c}$ $(a = c^b)$,
\item $\exists!{c}$ $(a = c_b)$,
\item $a_b = b \Leftrightarrow b^a = a$,
\item $(a_b)^{(b^a)} = a$,
\item $(a^b)_{(b_a)} = a$,
\item $(a^b)^c = (a^{c_b})^{b^c}$,
\item $(a_b)_c = (a_{c^b})_{b_c}$,
\item $(a_b)^{c_{b^a}} = (a^c)_{b^{c_a}}$,
\end{enumerate}
where the free variables in (1-8) are universally quantified. The
definition can be simplified.

\begin{thm}
Axioms 1, 2, 4, and 6 imply the rest.
\end{thm}

\begin{proof}
Proofproofproofproofproof
\end{proof}


%%%%%%%%%%%%%%%%%%%%%%%%%%%%%%%%%%%%%%%%%%%%%%%%%%%%%%%%%%%%%%%%%%%%%%
\section{Semiquandle Families}
%%%%%%%%%%%%%%%%%%%%%%%%%%%%%%%%%%%%%%%%%%%%%%%%%%%%%%%%%%%%%%%%%%%%%%

\subsection{Semiquandles of Constant Action}

\begin{thm}
Let $\varphi \in S_n$. The operations given by $a^{b} = \varphi(a)$
and $a_b = \varphi^{-1}(a)$ form a semiquandle on $\{0, 1, \ldots,
n-1\}$ which we denote by $C(\varphi)$.
\end{thm}

\begin{proof}
  Axioms (1) and (2) hold because the operations come from bijections.

For (4): $(a_b)^{(b^a)} = \phi(a_b) = \phi \phi^{-1}(a) = a$.

For (6): $(a^b)^c = \phi^{2}(a) = (a^{c_b})^{b^c}$.
\end{proof}

\begin{definition}
A semiquandle is \emph{of constant action} if it equals $C(\varphi)$
for some $\varphi$.
\end{definition}

\begin{thm}
All semiquandles smaller than size three are simple and of constant
action.
\end{thm}

\begin{proof}
Computer exhaustion reveals there are three semiquandles (up to
isomorphism) smaller than size three. They are of constant action.
There is no way to partition a set of size one or two other than
trivially, so the semiquandles are simple.
\end{proof}

We can relate properties of the cyclic group $\langle\varphi\rangle$
to those of the semiquandle $C(\varphi)$.  The following theorem
establishes that non-simplicity begets non-simplicity when $\varphi$
is a cycle.

\begin{thm}
Let $\varphi$ be an $n$-cycle in $S_n$. If $N \lhd \langle\varphi\rangle$
then there is a congruence on $C(\varphi)$ having
$\left|\langle\varphi\rangle : N\right|$ classes.
\end{thm}

\begin{proof}
Since $C(\varphi)$ is nonempty $0 \in C(\varphi)$. Since $\varphi$
is an $n$-cycle there is an $i,j$ such that $a = \varphi^{i}(0)$
and $b = \varphi^{j}(0)$ for any $a,b \in C(\varphi)$.  So let
$\sim$ be the relation on $C(\varphi)$ defined by $\varphi^{i}(0)
\sim \varphi^{j}(0)$ iff the cosets $N\varphi^{i} = N\varphi^{j}$.
First, $\sim$ is an equivalence relation by the properties of
equality. Now suppose some $\varphi^{i}(0) \sim \varphi^{j}(0)$.
Then by normality $N\varphi^{i+1} = N\varphi^{i}N\varphi =
N\varphi^{j}N\varphi = N\varphi^{j+1}$. Hence $a \sim b$ implies
$\varphi(a) \sim \varphi(b)$ and $\varphi^{-1}(a) \sim \varphi^{-1}(b)$
for any $a$, $b \in C(\varphi)$. Thus $\sim$ is a congruence.
\end{proof}

The other direction does not hold in general. Permutations which
generate simple groups don't always induce simple semiquandles of
constant action. Here is where the trouble occurs.

\begin{thm}
If $\varphi \in S_{n > 2}$ fixes an element then $C(\varphi)$ is
not simple.
\end{thm}

\begin{proof}
Suppose $\varphi$ fixes $x$ and consider the partition $\{ \{x\},
C(\varphi) \setminus \{x\} \}$.  If $a \sim b$ under the induced
equivalence relation then either both or neither equal $x$.  So
both or neither of $\varphi(a)$, $\varphi(b)$ equal $x$, making
$\varphi(a) \sim \varphi(b)$.  Hence the partition is a nontrivial
congruence.
\end{proof}

All simple semiquandles of constant action come from permutations
which generate simple groups and fix nothing.

\begin{thm}
Let $p$ be prime and let $\varphi$ be a p-cycle in the symmetric
group $S_p$.  Then $C(\varphi)$ is simple.
\end{thm}

\begin{proof}
Let $\Theta$ be a congruence on $C(\varphi)$. Suppose $\Theta$ has
a class with at least two elements, $a$ and $b$. We will show that
this class is all of $C(\varphi)$, and hence that $\Theta$ is
trivial.

Since $\varphi$ is a cycle, $b = \varphi^{i}(a)$ for some $i$; hence
$\varphi^{i}(a) \equiv \varphi^{0}(a)$. Since $\Theta$ is a congruence
$\varphi^{i}(a) = \varphi^{i}(\varphi^{0}(a)) \equiv
\varphi^{i}(\varphi^{i}(a)) = \varphi^{2i}(a)$. By transitivity,
$a \equiv \varphi^{2i}(a)$. Repeating the process gives $a \equiv
\varphi^{ni}(a)$ for any $n$.

If $c$ is an arbitrary element of $C(\varphi)$, then $c = \varphi^{j}(a)$
for some $j$.  Since the $\varphi$ has prime order in $S_p$, there
exists an $n$ with $\varphi^{j} = \varphi^{ni}$ so $a \equiv
\varphi^{ni}(a) = c$.  \end{proof}

\subsection{Latin Semiquandles}

A \emph{Latin} semiquandle is one which additionally satisfies
$\forall{ab}$ $\exists!{c}$ $(a=b^c)$ and $\forall{ab}$ $\exists!{c}$
$(a=b_c)$. We write $b/a$ for the unique $c$ such that $a = b^{c}$.

\begin{thm}
The $\nearrow$ and $\searrow$ operations differ in a nontrivial
Latin semiquandle.
\end{thm}
\begin{proof}
Assume for contradiction that $a^{b} = a_b$ for all $a$ and $b$.
Then Axioms 4 and 6 become $(x^y)^{(y^x)} = x$ and $(x^{(y^z)})^{(z^y)}
= (x^z)^y$ respectively.

We first show that this implies $y^{x}/x^{y} = x/y$. As a preliminary,
notice that $x = (x^{x/y})^{(x/y)^{x}} = y^{(x/y)^{x}}$. Hence
$(y^{x})^{y^{x}/x^{y}} = x^{y} = (y^{(x/y)^{x}})^{y} =
(y^{(x/y)^{x}})^{x^{x/y}} = (y^{x})^{x/y}$. By cancellation of the
$y^{x}$ base, $y^{x} / x^{y} = x/y$.

Next we'll see that $(x/y)/(y/x) = x$. First, $y^{{(x/y)}^{x}} =
(x^{x/y})^{{(x/y)}^{x}} = x = y^{y/x}$. By cancelling the $y$ base
we get $(x/y)^{x} = y/x$ which itself equals $(x/y)^{(x/y)/(y/x)}$.
Cancelling the $x/y$ base gives us the result.

Combining these facts yields $y^{x} = (y^{x}/x^{y}) / (x^{y}/y^{x})
= (x/y) / (y/x) = x$. However, if the semiquandle has two distinct
elements $a \neq b$ then $a^{x} = x = b^{x}$, which contradicts the
Latin condition.
\end{proof}

\begin{thebibliography}{9}

   \bibitem{aH10}
      A. Henrich and S. Nelson, \emph{Semiquandles and flat virtual
      knots}, Pacific J. Math. \textbf{248} (2010), no. 1, 155--170.
   \bibitem{unotdP9}
      W. McCune, ``Prover9 and Mace4'', http://www.cs.unm.edu/~mccune/Prover9, 2005-2010.

\end{thebibliography}

\end{document}
