\documentclass{amsart}
\newtheorem{thm}{Theorem}[section]
\newtheorem{prop}[thm]{Proposition}
\newtheorem{lem}[thm]{Lemma}
\newtheorem{cor}[thm]{Corollary}
\theoremstyle{definition}
\newtheorem{definition}[thm]{Definition}
\newtheorem{example}[thm]{Example}
\numberwithin{equation}{section}


\begin{document}


\title{Classification of Finite Symmetric Biquandles}

\author{Joe Nelson}
\author{Scott Pellicane}

\begin{abstract}
The goal is to classify finite symmetric biquandles, or semiquandles, first defined in \cite{aH10}.

We present an infinite class of simple semiquandles of prime order.
\end{abstract}

\maketitle

%\tableofcontents

%%%%%%%%%%%%%%%%%%%%%%%%%%%%%%%%%%%%%%%%%%%%%%%%%%%%%%%%%%%%%%%%%%%%%%
\section{Definition of a Semiquandle}
%%%%%%%%%%%%%%%%%%%%%%%%%%%%%%%%%%%%%%%%%%%%%%%%%%%%%%%%%%%%%%%%%%%%%%

A \emph{semiquandle} (as defined in \cite{aH10}) is a system
$(X,\nearrow,\searrow)$ such that $X$ is a nonempty set and $\nearrow$
and $\searrow$ are binary functions on $X$ satisfying:

\begin{enumerate}
\item $\exists!{c}$ $(a = c^b)$,
\item $\exists!{c}$ $(a = c_b)$,
\item $a_b = b \Leftrightarrow b^a = a$,
\item $(a_b)^{(b^a)} = a$,
\item $(a^b)_{(b_a)} = a$,
\item $(a^b)^c = (a^{c_b})^{b^c}$,
\item $(a_b)_c = (a_{c^b})_{b_c}$,
\item $(a_b)^{c_{b^a}} = (a^c)_{b^{c_a}}$,
\end{enumerate}
where the free variables in (1-8) are universally quantified. A
\emph{Latin semiquandle} additionally satisfies $\forall{ab}$
$\exists!{c}$ $(a=b^c)$ and $\forall{ab}$ $\exists!{c}$ $(a=b_c)$.

The definition can be simplified.

\begin{lem}
Axioms 1, 2, 4, and 6 imply the rest.
\end{lem}

\begin{proof}
Proofproofproofproofproof
\end{proof}


%%%%%%%%%%%%%%%%%%%%%%%%%%%%%%%%%%%%%%%%%%%%%%%%%%%%%%%%%%%%%%%%%%%%%%
\section{Semiquandle Families}
%%%%%%%%%%%%%%%%%%%%%%%%%%%%%%%%%%%%%%%%%%%%%%%%%%%%%%%%%%%%%%%%%%%%%%

\subsection{Semiquandles of Constant Action}

\begin{thm}
Let $\varphi \in S_n$. The operations given by $a^{b} = \varphi(a)$
and $a_b = \varphi^{-1}(a)$ form a semiquandle on $\{0, 1, \ldots,
n-1\}$ which we denote by $C(\varphi)$.
\end{thm}

\begin{proof}
  Axioms (1) and (2) hold because the operations come from bijections.

For (4): $(a_b)^{(b^a)} = \phi(a_b) = \phi \phi^{-1}(a) = a$.

For (6): $(a^b)^c = \phi^{2}(a) = (a^{c_b})^{b^c}$.

\end{proof}

\begin{definition}
A semiquandle is \emph{of constant action} if it equals $C(\varphi)$ for some $\varphi$.
\end{definition}

\begin{thm}
All semiquandles smaller than size three are simple and of constant action.
\end{thm}

\begin{proof}
Computer exhaustion reveals there are three semiquandles (up to isomorphism) smaller than
size three. They are of constant action. There is no way to partition a set of size one or two
other than trivially, so the semiquandles are simple.
\end{proof}

We can relate properties of the group $\langle\varphi\rangle$ to those of the semiquandle $C(\varphi)$.
The following theorem establishes that non-simplicity begets non-simplicity.

\begin{thm}
Let $\varphi \in S_n$. If $N \lhd \langle\varphi\rangle$ then there is a congruence on
$C(\varphi)$ having $\left|\langle\varphi\rangle : N\right|$ classes.
\end{thm}

\begin{proof}
Define a relation $\sim$ on $C(\varphi)$ by $\varphi^{i}(0) \sim \varphi^{j}(0)$ iff the cosets $N\varphi^{i} = N\varphi^{j}$. $\sim$ is an equivalence relation. Now suppose some $\varphi^{i}(0) \sim \varphi^{j}(0)$. Then by normality $N\varphi^{i+1} = N\varphi^{i}N\varphi = N\varphi^{j}N\varphi = N\varphi^{j+1}$. Hence $a \sim b$ implies $\varphi(a) \sim \varphi(b)$ and $\varphi^{-1}(a) \sim \varphi^{-1}(b)$ for any $a$, $b \in C(\varphi)$. Thus $\sim$ is a congruence.

[TODO: why are the congruences distinct?]

\bigskip

Let $B_1$, $B_2$ be blocks of a partition $\theta \in Con(C(\varphi))$ such that $n > \left|B_1\right| > \left|B_2\right| > 1$. Goal: $\left|B_1\right| = \left|B_2\right|$.

\bigskip

First, if any block $B$ of $\theta$ is closed (as an element of the semiquandle quotient structure), then $B$ contains all elements of $C(\varphi))$: $\varphi^k(\varphi^i(0)) \in B$ for all $k$.

\bigskip


\end{proof}

The other direction does not hold in general. Permutations which generate simple groups don't
always induce simple semiquandles of constant action. Here is where the trouble occurs.

\begin{thm}
If $\varphi \in S_{n > 2}$ fixes an element then $C(\varphi)$ is not simple.
\end{thm}

\begin{proof}
Suppose $\varphi$ fixes $x$ and consider the partition $\{ \{x\}, C(\varphi) \setminus \{x\} \}$.
If $a \sim b$ under the induced equivalence relation then either both or neither equal $x$.
So both or neither of $\varphi(a)$, $\varphi(b)$ equal $x$, making $\varphi(a) \sim \varphi(b)$.
Hence the partition is a nontrivial congruence.
\end{proof}

All simple semiquandles of constant action come from permutations
which generate simple groups and fix nothing.

\begin{thm}
Let $p$ be prime and let $\varphi$ be a p-cycle in the symmetric group $S_p$.
Then $C(\varphi)$ is simple.
\end{thm}

\begin{proof}
Let $\Theta$ be a congruence on $C(\varphi)$. Suppose $\Theta$ has
a class with at least two elements, $a$ and $b$. We will show that
this class is all of $C(\varphi)$, and hence that $\Theta$ is
trivial.

Since $\varphi$ is a cycle, $b = \varphi^{i}(a)$ for some $i$; hence
$\varphi^{i}(a) \equiv \varphi^{0}(a)$. Since $\Theta$ is a congruence
$\varphi^{i}(a) = \varphi^{i}(\varphi^{0}(a)) \equiv
\varphi^{i}(\varphi^{i}(a)) = \varphi^{2i}(a)$. By transitivity,
$a \equiv \varphi^{2i}(a)$. Repeating the process gives $a \equiv
\varphi^{ni}(a)$ for any $n$.

If $c$ is an arbitrary element of $C(\varphi)$, then $c = \varphi^{j}(a)$
for some $j$.  Since the $\varphi$ has prime order in $S_p$, there
exists an $n$ with $\varphi^{j} = \varphi^{ni}$ so $a \equiv
\varphi^{ni}(a) = c$.
\end{proof}

\subsection{Latin Semiquandles}


\begin{thebibliography}{9}

   \bibitem{aH10}
      A. Henrich and S. Nelson, \emph{Semiquandles and flat virtual
      knots}, Pacific J. Math. \textbf{248} (2010), no. 1, 155--170.

\end{thebibliography}

\end{document}
