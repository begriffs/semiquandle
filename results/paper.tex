\documentclass{amsart}
\newtheorem{thm}{Theorem}[section]
\newtheorem{prop}[thm]{Proposition}
\newtheorem{lem}[thm]{Lemma}
\newtheorem{cor}[thm]{Corollary}
\theoremstyle{definition}
\newtheorem{definition}[thm]{Definition}
\newtheorem{example}[thm]{Example}
\numberwithin{equation}{section}


\begin{document}


\title{Classification of Finite Symmetric Biquandles}

\author{Joe Nelson}
\email{joe@begriffs.com}
\urladdr{https://github.com/begriffs/semiquandle}
\author{Scott Pellicane}
\email{second@math.sc.edu}


\begin{abstract}
The goal is to classify finite symmetric biquandles, or semiquandles, first defined in [1].

We present an infinite class of simple semiquandles of prime order.
\end{abstract}

\maketitle

%\tableofcontents


%%%%%%%%%%%%%%%%%%%%%%%%%%%%%%%%%%%%%%%%%%%%%%%%%%%%%%%%%%%%%%%%%%%%%%
\section{Semiquandle Families}
%%%%%%%%%%%%%%%%%%%%%%%%%%%%%%%%%%%%%%%%%%%%%%%%%%%%%%%%%%%%%%%%%%%%%%

\subsection{Semiquandles of Constant Action}

\begin{thm}
Let $\phi \in S_n$. The operations given by $a^{b} = \phi(a)$ and $a_b = \phi^{-1}(a)$ form
a semiquandle on $\{0, 1, \ldots, n-1\}$ which we denote $C(\phi)$.
\end{thm}
\begin{proof}
See [1].
\end{proof}

\begin{definition}
A semiquandle is \emph{of constant action} if it equals $C(\phi)$ for some $\phi$.
\end{definition}

\begin{thm}
All semiquandles smaller than size three are simple and of constant action.
\end{thm}
\begin{proof}
Computer exhaustion reveals there are three semiquandles (up to isomorphism) smaller than
size three. They are of constant action. There is no way to partition a set of size one or two
other than trivially, so the semiquandles are simple.
\end{proof}

We can relate properties of the group $\langle\phi\rangle$ to those of the semiquandle $C(\phi)$.
The following theorem establishes that non-simplicity begets non-simplicity.

\begin{thm}
Let $\phi \in S_n$. If $N \lhd \langle\phi\rangle$ then there is a congruence on
$C(\phi)$ having $\left|\langle\phi\rangle : N\right|$ classes.
\end{thm}
\begin{proof}
Define a relation $\sim$ on $C(\phi)$ by $\phi^{i}(0) \sim \phi^{j}(0)$ iff
the cosets $N\phi^{i} = N\phi^{j}$. Now suppose some $\phi^{i}(0) \sim \phi^{j}(0)$. Then
$N\phi^{i+1} = N\phi^{i}N\phi = N\phi^{j}N\phi = N\phi^{j+1}$, the outermost equalities holding
by normality. Hence $a \sim b$ implies both $\phi(a) \sim \phi(b)$ for any $a$, $b \in C(\phi)$
and similarly $\phi^{-1}(a) \sim \phi^{-1}(b)$. Thus $\sim$ is a congruence.

[TODO: why are the congruences distinct?]
\end{proof}

The other direction does not hold in general. Permutations which generate simple groups don't
always induce simple semiquandles of constant action. Here is where the trouble occurs.

\begin{thm}
If $\phi$ fixes an element then $C(\phi)$ is not simple.
\end{thm}
\begin{proof}
Suppose $\phi$ fixes $x$ and consider the partition $\{ \{x\}, C(\phi) \setminus \{x\} \}$.
If $a \sim b$ under the induced equivalence relation then either both or neither equal $x$.
So both or neither of $\phi(a)$, $\phi(b)$ equal $x$, making $\phi(a) \sim \phi(b)$.
Hence the partition is a nontrivial congruence.
\end{proof}

The following shows the source of all simple semiquandles of constant action: permutations
which generate simple groups and which fix nothing.

\begin{thm}
Let $p$ be prime and let $\phi$ be a p-cycle in the symmetric group $S_p$.
Then $C(\phi)$ is simple.
\end{thm}
\begin{proof}
Let $\Theta$ be a congruence on $C(\phi)$. Suppose $\Theta$ has a class with
at least two elements, $a$ and $b$. We will show that this class is all of
$C(\phi)$, and hence that $\Theta$ is trivial.

Since $\phi$ is a cycle, $b = \phi^{i}(a)$ for some $i$; hence
$\phi^{i}(a) \equiv \phi^{0}(a)$. Since $\Theta$ is a congruence $\phi^{i}(a) =
\phi^{i}(\phi^{0}(a)) \equiv \phi^{i}(\phi^{i}(a)) = \phi^{2i}(a)$. By transitivity,
$a \equiv \phi^{2i}(a)$. Repeating the process gives $a \equiv \phi^{ni}(a)$ for any $n$.

If $c$ is an arbitrary element of $C(\phi)$, then $c = \phi^{j}(a)$ for some $j$.
Since the $\phi$ has prime order in $S_p$, there exists an $n$ with
$\phi^{j} = \phi^{ni}$ so $a \equiv \phi^{ni}(a) = c$.
\end{proof}

\subsection{Latin Semiquandles}

\end{document}
